
% Default to the notebook output style

    


% Inherit from the specified cell style.




    
\documentclass[11pt]{article}

    
    
    \usepackage[T1]{fontenc}
    % Nicer default font (+ math font) than Computer Modern for most use cases
    \usepackage{mathpazo}

    % Basic figure setup, for now with no caption control since it's done
    % automatically by Pandoc (which extracts ![](path) syntax from Markdown).
    \usepackage{graphicx}
    % We will generate all images so they have a width \maxwidth. This means
    % that they will get their normal width if they fit onto the page, but
    % are scaled down if they would overflow the margins.
    \makeatletter
    \def\maxwidth{\ifdim\Gin@nat@width>\linewidth\linewidth
    \else\Gin@nat@width\fi}
    \makeatother
    \let\Oldincludegraphics\includegraphics
    % Set max figure width to be 80% of text width, for now hardcoded.
    \renewcommand{\includegraphics}[1]{\Oldincludegraphics[width=.8\maxwidth]{#1}}
    % Ensure that by default, figures have no caption (until we provide a
    % proper Figure object with a Caption API and a way to capture that
    % in the conversion process - todo).
    \usepackage{caption}
    \DeclareCaptionLabelFormat{nolabel}{}
    \captionsetup{labelformat=nolabel}

    \usepackage{adjustbox} % Used to constrain images to a maximum size 
    \usepackage{xcolor} % Allow colors to be defined
    \usepackage{enumerate} % Needed for markdown enumerations to work
    \usepackage{geometry} % Used to adjust the document margins
    \usepackage{amsmath} % Equations
    \usepackage{amssymb} % Equations
    \usepackage{textcomp} % defines textquotesingle
    % Hack from http://tex.stackexchange.com/a/47451/13684:
    \AtBeginDocument{%
        \def\PYZsq{\textquotesingle}% Upright quotes in Pygmentized code
    }
    \usepackage{upquote} % Upright quotes for verbatim code
    \usepackage{eurosym} % defines \euro
    \usepackage[mathletters]{ucs} % Extended unicode (utf-8) support
    \usepackage[utf8x]{inputenc} % Allow utf-8 characters in the tex document
    \usepackage{fancyvrb} % verbatim replacement that allows latex
    \usepackage{grffile} % extends the file name processing of package graphics 
                         % to support a larger range 
    % The hyperref package gives us a pdf with properly built
    % internal navigation ('pdf bookmarks' for the table of contents,
    % internal cross-reference links, web links for URLs, etc.)
    \usepackage{hyperref}
    \usepackage{longtable} % longtable support required by pandoc >1.10
    \usepackage{booktabs}  % table support for pandoc > 1.12.2
    \usepackage[inline]{enumitem} % IRkernel/repr support (it uses the enumerate* environment)
    \usepackage[normalem]{ulem} % ulem is needed to support strikethroughs (\sout)
                                % normalem makes italics be italics, not underlines
    

    
    
    % Colors for the hyperref package
    \definecolor{urlcolor}{rgb}{0,.145,.698}
    \definecolor{linkcolor}{rgb}{.71,0.21,0.01}
    \definecolor{citecolor}{rgb}{.12,.54,.11}

    % ANSI colors
    \definecolor{ansi-black}{HTML}{3E424D}
    \definecolor{ansi-black-intense}{HTML}{282C36}
    \definecolor{ansi-red}{HTML}{E75C58}
    \definecolor{ansi-red-intense}{HTML}{B22B31}
    \definecolor{ansi-green}{HTML}{00A250}
    \definecolor{ansi-green-intense}{HTML}{007427}
    \definecolor{ansi-yellow}{HTML}{DDB62B}
    \definecolor{ansi-yellow-intense}{HTML}{B27D12}
    \definecolor{ansi-blue}{HTML}{208FFB}
    \definecolor{ansi-blue-intense}{HTML}{0065CA}
    \definecolor{ansi-magenta}{HTML}{D160C4}
    \definecolor{ansi-magenta-intense}{HTML}{A03196}
    \definecolor{ansi-cyan}{HTML}{60C6C8}
    \definecolor{ansi-cyan-intense}{HTML}{258F8F}
    \definecolor{ansi-white}{HTML}{C5C1B4}
    \definecolor{ansi-white-intense}{HTML}{A1A6B2}

    % commands and environments needed by pandoc snippets
    % extracted from the output of `pandoc -s`
    \providecommand{\tightlist}{%
      \setlength{\itemsep}{0pt}\setlength{\parskip}{0pt}}
    \DefineVerbatimEnvironment{Highlighting}{Verbatim}{commandchars=\\\{\}}
    % Add ',fontsize=\small' for more characters per line
    \newenvironment{Shaded}{}{}
    \newcommand{\KeywordTok}[1]{\textcolor[rgb]{0.00,0.44,0.13}{\textbf{{#1}}}}
    \newcommand{\DataTypeTok}[1]{\textcolor[rgb]{0.56,0.13,0.00}{{#1}}}
    \newcommand{\DecValTok}[1]{\textcolor[rgb]{0.25,0.63,0.44}{{#1}}}
    \newcommand{\BaseNTok}[1]{\textcolor[rgb]{0.25,0.63,0.44}{{#1}}}
    \newcommand{\FloatTok}[1]{\textcolor[rgb]{0.25,0.63,0.44}{{#1}}}
    \newcommand{\CharTok}[1]{\textcolor[rgb]{0.25,0.44,0.63}{{#1}}}
    \newcommand{\StringTok}[1]{\textcolor[rgb]{0.25,0.44,0.63}{{#1}}}
    \newcommand{\CommentTok}[1]{\textcolor[rgb]{0.38,0.63,0.69}{\textit{{#1}}}}
    \newcommand{\OtherTok}[1]{\textcolor[rgb]{0.00,0.44,0.13}{{#1}}}
    \newcommand{\AlertTok}[1]{\textcolor[rgb]{1.00,0.00,0.00}{\textbf{{#1}}}}
    \newcommand{\FunctionTok}[1]{\textcolor[rgb]{0.02,0.16,0.49}{{#1}}}
    \newcommand{\RegionMarkerTok}[1]{{#1}}
    \newcommand{\ErrorTok}[1]{\textcolor[rgb]{1.00,0.00,0.00}{\textbf{{#1}}}}
    \newcommand{\NormalTok}[1]{{#1}}
    
    % Additional commands for more recent versions of Pandoc
    \newcommand{\ConstantTok}[1]{\textcolor[rgb]{0.53,0.00,0.00}{{#1}}}
    \newcommand{\SpecialCharTok}[1]{\textcolor[rgb]{0.25,0.44,0.63}{{#1}}}
    \newcommand{\VerbatimStringTok}[1]{\textcolor[rgb]{0.25,0.44,0.63}{{#1}}}
    \newcommand{\SpecialStringTok}[1]{\textcolor[rgb]{0.73,0.40,0.53}{{#1}}}
    \newcommand{\ImportTok}[1]{{#1}}
    \newcommand{\DocumentationTok}[1]{\textcolor[rgb]{0.73,0.13,0.13}{\textit{{#1}}}}
    \newcommand{\AnnotationTok}[1]{\textcolor[rgb]{0.38,0.63,0.69}{\textbf{\textit{{#1}}}}}
    \newcommand{\CommentVarTok}[1]{\textcolor[rgb]{0.38,0.63,0.69}{\textbf{\textit{{#1}}}}}
    \newcommand{\VariableTok}[1]{\textcolor[rgb]{0.10,0.09,0.49}{{#1}}}
    \newcommand{\ControlFlowTok}[1]{\textcolor[rgb]{0.00,0.44,0.13}{\textbf{{#1}}}}
    \newcommand{\OperatorTok}[1]{\textcolor[rgb]{0.40,0.40,0.40}{{#1}}}
    \newcommand{\BuiltInTok}[1]{{#1}}
    \newcommand{\ExtensionTok}[1]{{#1}}
    \newcommand{\PreprocessorTok}[1]{\textcolor[rgb]{0.74,0.48,0.00}{{#1}}}
    \newcommand{\AttributeTok}[1]{\textcolor[rgb]{0.49,0.56,0.16}{{#1}}}
    \newcommand{\InformationTok}[1]{\textcolor[rgb]{0.38,0.63,0.69}{\textbf{\textit{{#1}}}}}
    \newcommand{\WarningTok}[1]{\textcolor[rgb]{0.38,0.63,0.69}{\textbf{\textit{{#1}}}}}
    
    
    % Define a nice break command that doesn't care if a line doesn't already
    % exist.
    \def\br{\hspace*{\fill} \\* }
    % Math Jax compatability definitions
    \def\gt{>}
    \def\lt{<}
    % Document parameters
    \title{Untitled}
    
    
    

    % Pygments definitions
    
\makeatletter
\def\PY@reset{\let\PY@it=\relax \let\PY@bf=\relax%
    \let\PY@ul=\relax \let\PY@tc=\relax%
    \let\PY@bc=\relax \let\PY@ff=\relax}
\def\PY@tok#1{\csname PY@tok@#1\endcsname}
\def\PY@toks#1+{\ifx\relax#1\empty\else%
    \PY@tok{#1}\expandafter\PY@toks\fi}
\def\PY@do#1{\PY@bc{\PY@tc{\PY@ul{%
    \PY@it{\PY@bf{\PY@ff{#1}}}}}}}
\def\PY#1#2{\PY@reset\PY@toks#1+\relax+\PY@do{#2}}

\expandafter\def\csname PY@tok@w\endcsname{\def\PY@tc##1{\textcolor[rgb]{0.73,0.73,0.73}{##1}}}
\expandafter\def\csname PY@tok@c\endcsname{\let\PY@it=\textit\def\PY@tc##1{\textcolor[rgb]{0.25,0.50,0.50}{##1}}}
\expandafter\def\csname PY@tok@cp\endcsname{\def\PY@tc##1{\textcolor[rgb]{0.74,0.48,0.00}{##1}}}
\expandafter\def\csname PY@tok@k\endcsname{\let\PY@bf=\textbf\def\PY@tc##1{\textcolor[rgb]{0.00,0.50,0.00}{##1}}}
\expandafter\def\csname PY@tok@kp\endcsname{\def\PY@tc##1{\textcolor[rgb]{0.00,0.50,0.00}{##1}}}
\expandafter\def\csname PY@tok@kt\endcsname{\def\PY@tc##1{\textcolor[rgb]{0.69,0.00,0.25}{##1}}}
\expandafter\def\csname PY@tok@o\endcsname{\def\PY@tc##1{\textcolor[rgb]{0.40,0.40,0.40}{##1}}}
\expandafter\def\csname PY@tok@ow\endcsname{\let\PY@bf=\textbf\def\PY@tc##1{\textcolor[rgb]{0.67,0.13,1.00}{##1}}}
\expandafter\def\csname PY@tok@nb\endcsname{\def\PY@tc##1{\textcolor[rgb]{0.00,0.50,0.00}{##1}}}
\expandafter\def\csname PY@tok@nf\endcsname{\def\PY@tc##1{\textcolor[rgb]{0.00,0.00,1.00}{##1}}}
\expandafter\def\csname PY@tok@nc\endcsname{\let\PY@bf=\textbf\def\PY@tc##1{\textcolor[rgb]{0.00,0.00,1.00}{##1}}}
\expandafter\def\csname PY@tok@nn\endcsname{\let\PY@bf=\textbf\def\PY@tc##1{\textcolor[rgb]{0.00,0.00,1.00}{##1}}}
\expandafter\def\csname PY@tok@ne\endcsname{\let\PY@bf=\textbf\def\PY@tc##1{\textcolor[rgb]{0.82,0.25,0.23}{##1}}}
\expandafter\def\csname PY@tok@nv\endcsname{\def\PY@tc##1{\textcolor[rgb]{0.10,0.09,0.49}{##1}}}
\expandafter\def\csname PY@tok@no\endcsname{\def\PY@tc##1{\textcolor[rgb]{0.53,0.00,0.00}{##1}}}
\expandafter\def\csname PY@tok@nl\endcsname{\def\PY@tc##1{\textcolor[rgb]{0.63,0.63,0.00}{##1}}}
\expandafter\def\csname PY@tok@ni\endcsname{\let\PY@bf=\textbf\def\PY@tc##1{\textcolor[rgb]{0.60,0.60,0.60}{##1}}}
\expandafter\def\csname PY@tok@na\endcsname{\def\PY@tc##1{\textcolor[rgb]{0.49,0.56,0.16}{##1}}}
\expandafter\def\csname PY@tok@nt\endcsname{\let\PY@bf=\textbf\def\PY@tc##1{\textcolor[rgb]{0.00,0.50,0.00}{##1}}}
\expandafter\def\csname PY@tok@nd\endcsname{\def\PY@tc##1{\textcolor[rgb]{0.67,0.13,1.00}{##1}}}
\expandafter\def\csname PY@tok@s\endcsname{\def\PY@tc##1{\textcolor[rgb]{0.73,0.13,0.13}{##1}}}
\expandafter\def\csname PY@tok@sd\endcsname{\let\PY@it=\textit\def\PY@tc##1{\textcolor[rgb]{0.73,0.13,0.13}{##1}}}
\expandafter\def\csname PY@tok@si\endcsname{\let\PY@bf=\textbf\def\PY@tc##1{\textcolor[rgb]{0.73,0.40,0.53}{##1}}}
\expandafter\def\csname PY@tok@se\endcsname{\let\PY@bf=\textbf\def\PY@tc##1{\textcolor[rgb]{0.73,0.40,0.13}{##1}}}
\expandafter\def\csname PY@tok@sr\endcsname{\def\PY@tc##1{\textcolor[rgb]{0.73,0.40,0.53}{##1}}}
\expandafter\def\csname PY@tok@ss\endcsname{\def\PY@tc##1{\textcolor[rgb]{0.10,0.09,0.49}{##1}}}
\expandafter\def\csname PY@tok@sx\endcsname{\def\PY@tc##1{\textcolor[rgb]{0.00,0.50,0.00}{##1}}}
\expandafter\def\csname PY@tok@m\endcsname{\def\PY@tc##1{\textcolor[rgb]{0.40,0.40,0.40}{##1}}}
\expandafter\def\csname PY@tok@gh\endcsname{\let\PY@bf=\textbf\def\PY@tc##1{\textcolor[rgb]{0.00,0.00,0.50}{##1}}}
\expandafter\def\csname PY@tok@gu\endcsname{\let\PY@bf=\textbf\def\PY@tc##1{\textcolor[rgb]{0.50,0.00,0.50}{##1}}}
\expandafter\def\csname PY@tok@gd\endcsname{\def\PY@tc##1{\textcolor[rgb]{0.63,0.00,0.00}{##1}}}
\expandafter\def\csname PY@tok@gi\endcsname{\def\PY@tc##1{\textcolor[rgb]{0.00,0.63,0.00}{##1}}}
\expandafter\def\csname PY@tok@gr\endcsname{\def\PY@tc##1{\textcolor[rgb]{1.00,0.00,0.00}{##1}}}
\expandafter\def\csname PY@tok@ge\endcsname{\let\PY@it=\textit}
\expandafter\def\csname PY@tok@gs\endcsname{\let\PY@bf=\textbf}
\expandafter\def\csname PY@tok@gp\endcsname{\let\PY@bf=\textbf\def\PY@tc##1{\textcolor[rgb]{0.00,0.00,0.50}{##1}}}
\expandafter\def\csname PY@tok@go\endcsname{\def\PY@tc##1{\textcolor[rgb]{0.53,0.53,0.53}{##1}}}
\expandafter\def\csname PY@tok@gt\endcsname{\def\PY@tc##1{\textcolor[rgb]{0.00,0.27,0.87}{##1}}}
\expandafter\def\csname PY@tok@err\endcsname{\def\PY@bc##1{\setlength{\fboxsep}{0pt}\fcolorbox[rgb]{1.00,0.00,0.00}{1,1,1}{\strut ##1}}}
\expandafter\def\csname PY@tok@kc\endcsname{\let\PY@bf=\textbf\def\PY@tc##1{\textcolor[rgb]{0.00,0.50,0.00}{##1}}}
\expandafter\def\csname PY@tok@kd\endcsname{\let\PY@bf=\textbf\def\PY@tc##1{\textcolor[rgb]{0.00,0.50,0.00}{##1}}}
\expandafter\def\csname PY@tok@kn\endcsname{\let\PY@bf=\textbf\def\PY@tc##1{\textcolor[rgb]{0.00,0.50,0.00}{##1}}}
\expandafter\def\csname PY@tok@kr\endcsname{\let\PY@bf=\textbf\def\PY@tc##1{\textcolor[rgb]{0.00,0.50,0.00}{##1}}}
\expandafter\def\csname PY@tok@bp\endcsname{\def\PY@tc##1{\textcolor[rgb]{0.00,0.50,0.00}{##1}}}
\expandafter\def\csname PY@tok@fm\endcsname{\def\PY@tc##1{\textcolor[rgb]{0.00,0.00,1.00}{##1}}}
\expandafter\def\csname PY@tok@vc\endcsname{\def\PY@tc##1{\textcolor[rgb]{0.10,0.09,0.49}{##1}}}
\expandafter\def\csname PY@tok@vg\endcsname{\def\PY@tc##1{\textcolor[rgb]{0.10,0.09,0.49}{##1}}}
\expandafter\def\csname PY@tok@vi\endcsname{\def\PY@tc##1{\textcolor[rgb]{0.10,0.09,0.49}{##1}}}
\expandafter\def\csname PY@tok@vm\endcsname{\def\PY@tc##1{\textcolor[rgb]{0.10,0.09,0.49}{##1}}}
\expandafter\def\csname PY@tok@sa\endcsname{\def\PY@tc##1{\textcolor[rgb]{0.73,0.13,0.13}{##1}}}
\expandafter\def\csname PY@tok@sb\endcsname{\def\PY@tc##1{\textcolor[rgb]{0.73,0.13,0.13}{##1}}}
\expandafter\def\csname PY@tok@sc\endcsname{\def\PY@tc##1{\textcolor[rgb]{0.73,0.13,0.13}{##1}}}
\expandafter\def\csname PY@tok@dl\endcsname{\def\PY@tc##1{\textcolor[rgb]{0.73,0.13,0.13}{##1}}}
\expandafter\def\csname PY@tok@s2\endcsname{\def\PY@tc##1{\textcolor[rgb]{0.73,0.13,0.13}{##1}}}
\expandafter\def\csname PY@tok@sh\endcsname{\def\PY@tc##1{\textcolor[rgb]{0.73,0.13,0.13}{##1}}}
\expandafter\def\csname PY@tok@s1\endcsname{\def\PY@tc##1{\textcolor[rgb]{0.73,0.13,0.13}{##1}}}
\expandafter\def\csname PY@tok@mb\endcsname{\def\PY@tc##1{\textcolor[rgb]{0.40,0.40,0.40}{##1}}}
\expandafter\def\csname PY@tok@mf\endcsname{\def\PY@tc##1{\textcolor[rgb]{0.40,0.40,0.40}{##1}}}
\expandafter\def\csname PY@tok@mh\endcsname{\def\PY@tc##1{\textcolor[rgb]{0.40,0.40,0.40}{##1}}}
\expandafter\def\csname PY@tok@mi\endcsname{\def\PY@tc##1{\textcolor[rgb]{0.40,0.40,0.40}{##1}}}
\expandafter\def\csname PY@tok@il\endcsname{\def\PY@tc##1{\textcolor[rgb]{0.40,0.40,0.40}{##1}}}
\expandafter\def\csname PY@tok@mo\endcsname{\def\PY@tc##1{\textcolor[rgb]{0.40,0.40,0.40}{##1}}}
\expandafter\def\csname PY@tok@ch\endcsname{\let\PY@it=\textit\def\PY@tc##1{\textcolor[rgb]{0.25,0.50,0.50}{##1}}}
\expandafter\def\csname PY@tok@cm\endcsname{\let\PY@it=\textit\def\PY@tc##1{\textcolor[rgb]{0.25,0.50,0.50}{##1}}}
\expandafter\def\csname PY@tok@cpf\endcsname{\let\PY@it=\textit\def\PY@tc##1{\textcolor[rgb]{0.25,0.50,0.50}{##1}}}
\expandafter\def\csname PY@tok@c1\endcsname{\let\PY@it=\textit\def\PY@tc##1{\textcolor[rgb]{0.25,0.50,0.50}{##1}}}
\expandafter\def\csname PY@tok@cs\endcsname{\let\PY@it=\textit\def\PY@tc##1{\textcolor[rgb]{0.25,0.50,0.50}{##1}}}

\def\PYZbs{\char`\\}
\def\PYZus{\char`\_}
\def\PYZob{\char`\{}
\def\PYZcb{\char`\}}
\def\PYZca{\char`\^}
\def\PYZam{\char`\&}
\def\PYZlt{\char`\<}
\def\PYZgt{\char`\>}
\def\PYZsh{\char`\#}
\def\PYZpc{\char`\%}
\def\PYZdl{\char`\$}
\def\PYZhy{\char`\-}
\def\PYZsq{\char`\'}
\def\PYZdq{\char`\"}
\def\PYZti{\char`\~}
% for compatibility with earlier versions
\def\PYZat{@}
\def\PYZlb{[}
\def\PYZrb{]}
\makeatother


    % Exact colors from NB
    \definecolor{incolor}{rgb}{0.0, 0.0, 0.5}
    \definecolor{outcolor}{rgb}{0.545, 0.0, 0.0}



    
    % Prevent overflowing lines due to hard-to-break entities
    \sloppy 
    % Setup hyperref package
    \hypersetup{
      breaklinks=true,  % so long urls are correctly broken across lines
      colorlinks=true,
      urlcolor=urlcolor,
      linkcolor=linkcolor,
      citecolor=citecolor,
      }
    % Slightly bigger margins than the latex defaults
    
    \geometry{verbose,tmargin=1in,bmargin=1in,lmargin=1in,rmargin=1in}
    
    

    \begin{document}
    
    
    \maketitle
    
    

    
    \hypertarget{smd-uxfcbungunszettel-nr.-7}{%
\section{SMD Übungunszettel Nr. 7}\label{smd-uxfcbungunszettel-nr.-7}}

Abgabe von: \textbf{Pape, Kusurmann und Becker}

    \begin{Verbatim}[commandchars=\\\{\}]
{\color{incolor}In [{\color{incolor}1}]:} \PY{k+kn}{import} \PY{n+nn}{numpy} \PY{k}{as} \PY{n+nn}{np} 
        \PY{k+kn}{import} \PY{n+nn}{matplotlib}\PY{n+nn}{.}\PY{n+nn}{pyplot} \PY{k}{as} \PY{n+nn}{plt}
        \PY{k+kn}{import} \PY{n+nn}{pandas} \PY{k}{as} \PY{n+nn}{pd}
        \PY{k+kn}{from} \PY{n+nn}{collections} \PY{k}{import} \PY{n}{Counter}
        
        \PY{n}{np}\PY{o}{.}\PY{n}{random}\PY{o}{.}\PY{n}{seed}\PY{p}{(}\PY{l+m+mi}{43}\PY{p}{)}
\end{Verbatim}


    \hypertarget{aufgabe-19}{%
\subsection{Aufgabe 19}\label{aufgabe-19}}

\hypertarget{aufgabenteil-a}{%
\subsubsection{Aufgabenteil a)}\label{aufgabenteil-a}}

Die Startmittelpunkte sind: \[
 \vec{m}_1 = \begin{pmatrix} 3 \\ 4 \end{pmatrix}, \quad \vec{m}_2 = \begin{pmatrix} 3 \\ 7 \end{pmatrix}, \quad \vec{m}_3 = \begin{pmatrix} 7 \\ 4 \end{pmatrix}
\] Überprüfe zu welchen Mittelpunkt der Punkt
\(\vec{x} = \begin{pmatrix} 5 \\ 1 \end{pmatrix}\) gehöhrt:
\[\begin{aligned}
\vec{d}_1 &= \begin{pmatrix} 3 \\ 4 \end{pmatrix} -  \begin{pmatrix} 5 \\ 1 \end{pmatrix} =  \begin{pmatrix} -2 \\ 3 \end{pmatrix}, \quad \left| \vec{d}_1 \right| = \sqrt{13} \\
\vec{d}_3 &= \begin{pmatrix} 7 \\ 4 \end{pmatrix} -  \begin{pmatrix} 5 \\ 1 \end{pmatrix} =  \begin{pmatrix} 2 \\ 3 \end{pmatrix}, \quad \left| \vec{d}_1 \right| = \sqrt{13} = \left| \vec{d}_1 \right|
\end{aligned}\]

Beide Mittelpunkte besitzen den gleichen Abstand zum Punkt \(\vec{x}\).
Aus diesem Grund packe ich den Punkt zur Menge \(3\).

Berechnung der neuen Mittelpunkte: \[ \begin{aligned}
\vec{m}_1 = \frac{1}{5} \begin{pmatrix} 12 \\ 15 \end{pmatrix}, \quad 
\vec{m}_2 = \begin{pmatrix} 1 \\ 4 \end{pmatrix}, \quad 
\vec{m}_3 = \frac{1}{6} \begin{pmatrix} 41 \\ 21 \end{pmatrix} 
\end{aligned} \]

    \hypertarget{aufgabenteil-b}{%
\subsubsection{Aufgabenteil b)}\label{aufgabenteil-b}}

\textbf{Zweiter Iterationsschritt}

Überprüfe zu welchen Mittelpunkt der Punkt
\(\vec{x} = \begin{pmatrix} 1 \\ 5 \end{pmatrix}\) gehöhrt:

\[\begin{aligned}
\vec{d}_1 &= \frac{1}{5}\begin{pmatrix} 12 \\ 5 \end{pmatrix} -  \begin{pmatrix} 1 \\ 5 \end{pmatrix} =  \frac{1}{5} \begin{pmatrix} 12 -5 \\ 15 - 25  \end{pmatrix} = \frac{1}{5}\begin{pmatrix} 7 \\ -10 \end{pmatrix}, \quad \left| \vec{d}_1 \right| = \sqrt{\frac{49}{25} + 4}  \\
\vec{d}_2 &= \begin{pmatrix} 1 \\ 6 \end{pmatrix} -  \begin{pmatrix} 1 \\ 5 \end{pmatrix} =  \begin{pmatrix} 0 \\ 1 \end{pmatrix}, \quad \left| \vec{d}_2 \right| = 1 < \left| \vec{d}_1 \right|
\end{aligned}\]

Überprüfe zu welchen Mittelpunkt der Punkt
\(\vec{x} = \begin{pmatrix} 1 \\ 4 \end{pmatrix}\) gehöhrt:

\[\begin{aligned}
\vec{d}_1 &= \frac{1}{5}\begin{pmatrix} 12 \\ 5 \end{pmatrix} -  \begin{pmatrix} 1 \\ 4 \end{pmatrix} =  \frac{1}{5} \begin{pmatrix} 12 -5 \\ 5 - 20  \end{pmatrix} = \frac{1}{5}\begin{pmatrix} 7 \\ -5 \end{pmatrix}, \quad \left| \vec{d}_1 \right| = \sqrt{\frac{49}{25} + 1}  \\
\vec{d}_2 &= \begin{pmatrix} 1 \\ 6 \end{pmatrix} -  \begin{pmatrix} 1 \\ 4 \end{pmatrix} =  \begin{pmatrix} 0 \\ 2 \end{pmatrix}, \quad \left| \vec{d}_2 \right| = 2 < \left| \vec{d}_1 \right|
\end{aligned}\]

Überprüfe zu welchen Mittelpunkt der Punkt
\(\vec{x} = \begin{pmatrix} 5 \\ 1 \end{pmatrix}\) gehöhrt:

\[\begin{aligned}
\vec{d}_1 &= \frac{1}{5}\begin{pmatrix} 12 \\ 5 \end{pmatrix} -  \begin{pmatrix} 5 \\ 1 \end{pmatrix} =  \frac{1}{5} \begin{pmatrix} 12 -25 \\ 15 - 5  \end{pmatrix} = \frac{1}{5}\begin{pmatrix} -13 \\ 10 \end{pmatrix}, \quad \left| \vec{d}_1 \right| = \sqrt{\frac{169}{25} + 4}  \\
\vec{d}_3 &= \frac{1}{6}\begin{pmatrix} 41 \\ 21 \end{pmatrix} -  \begin{pmatrix} 5 \\ 1 \end{pmatrix} =  \frac{1}{6} \begin{pmatrix} 41 -30 \\ 21 -6  \end{pmatrix} = \frac{1}{6}\begin{pmatrix} 11 \\ 15 \end{pmatrix}, \quad \left| \vec{d}_3 \right| = \sqrt{\frac{121}{36} + \frac{225}{36}} >  \left| \vec{d}_1 \right|  \\
\end{aligned}\]

Berechne die neuen Mittelpunkte:

\[ \begin{aligned}
\vec{m}_1 = \frac{1}{4} \begin{pmatrix} 11 \\ 10\end{pmatrix}, \quad 
\vec{m}_2 = \frac{1}{2} \begin{pmatrix} 2 \\ 11 \end{pmatrix}, \quad 
\vec{m}_3 = \frac{1}{6} \begin{pmatrix} 41 \\ 21 \end{pmatrix} 
\end{aligned} \]

\textbf{Dritter Iterationsschritt}

Überprüfe zu welchen Mittelpunkt der Punkt
\(\vec{x} = \begin{pmatrix} 1 \\ 4 \end{pmatrix}\) gehöhrt:

\[\begin{aligned}
\vec{d}_1 &= \frac{1}{4}\begin{pmatrix} 11 \\ 10 \end{pmatrix} -  \begin{pmatrix} 1 \\ 4 \end{pmatrix} =  \frac{1}{4} \begin{pmatrix} 11 - 4 \\ 10 - 16  \end{pmatrix} = \frac{1}{4}\begin{pmatrix} 7 \\ -6 \end{pmatrix}, \quad \left| \vec{d}_1 \right| = \sqrt{\frac{49}{16} + \frac{36}{16}}  \\
\vec{d}_2 &= \frac{1}{2}\begin{pmatrix} 2 \\ 11 \end{pmatrix} -  \begin{pmatrix} 1 \\ 4 \end{pmatrix} =  \frac{1}{2}\begin{pmatrix} 0 \\ 3 \end{pmatrix}, \quad \left| \vec{d}_2 \right| = \frac{3}{2} < \left| \vec{d}_1 \right|
\end{aligned}\]

Überprüfe zu welchen Mittelpunkt der Punkt
\(\vec{x} = \begin{pmatrix} 5 \\ 1 \end{pmatrix}\) gehöhrt:

\[\begin{aligned}
\vec{d}_1 &= \frac{1}{4}\begin{pmatrix} 11 \\ 10 \end{pmatrix} -  \begin{pmatrix} 5 \\ 1 \end{pmatrix} =  \frac{1}{4} \begin{pmatrix} 11 - 20 \\ 10 - 16  \end{pmatrix} = \frac{1}{4}\begin{pmatrix} -9 \\ -6 \end{pmatrix}, \quad \left| \vec{d}_1 \right| = \sqrt{\frac{81}{16} + \frac{36}{16}}  \\
\vec{d}_3 &= \frac{1}{6}\begin{pmatrix} 41 \\ 21 \end{pmatrix} -  \begin{pmatrix} 5 \\ 1 \end{pmatrix} =  \frac{1}{6}\begin{pmatrix} 11 \\ 15 \end{pmatrix}, \quad \left| \vec{d}_3 \right| = \sqrt{\frac{121}{36} + \frac{225}{36}} > \left| \vec{d}_1 \right|
\end{aligned}\]

Berechne die neuen Mittelpunkte:

\[ \begin{aligned}
\vec{m}_1 =  \begin{pmatrix} 1 \\ 5 \end{pmatrix}, \quad 
\vec{m}_2 = \frac{1}{4} \begin{pmatrix} 15 \\ 7 \end{pmatrix}, \quad 
\vec{m}_3 = \frac{1}{5} \begin{pmatrix} 36 \\ 2 \end{pmatrix} 
\end{aligned} \]

\textbf{Vierter Iterationsschritt}

Überprüfe zu welchen Mittelpunkt der Punkt
\(\vec{x} = \begin{pmatrix} 6 \\ 2 \end{pmatrix}\) gehöhrt:

\[\begin{aligned}
\vec{d}_1 &= \frac{1}{4}\begin{pmatrix} 15 \\ 7 \end{pmatrix} -  \begin{pmatrix} 6 \\ 2 \end{pmatrix} =  \frac{1}{4} \begin{pmatrix} 15 - 24 \\ 7 -8  \end{pmatrix} = \frac{1}{4}\begin{pmatrix} -9 \\ -1 \end{pmatrix}, \quad \left| \vec{d}_1 \right| = \sqrt{\frac{81}{16} + \frac{1}{16}}  \\
\vec{d}_3 &= \frac{1}{5}\begin{pmatrix} 36 \\ 20 \end{pmatrix} -  \begin{pmatrix} 6 \\ 2 \end{pmatrix} =  \frac{1}{5}\begin{pmatrix} 6 \\ 10 \end{pmatrix}, \quad \left| \vec{d}_3 \right| = \sqrt{\frac{36}{25} + 4} > \left| \vec{d}_1 \right|
\end{aligned}\]

Berechne die neuen Mittelpunkte:

\[ \begin{aligned}
\vec{m}_1 = \frac{1}{5} \begin{pmatrix} 21 \\ 9 \end{pmatrix}, \quad 
\vec{m}_2 =  \begin{pmatrix} 1 \\ 5 \end{pmatrix}, \quad 
\vec{m}_3 = \frac{1}{2} \begin{pmatrix} 15 \\ 9 \end{pmatrix} 
\end{aligned} \]

\textbf{Fünfter Iterationsschritt}

Überprüfe zu welchen Mittelpunkt der Punkt
\(\vec{x} = \begin{pmatrix} 6 \\ 3 \end{pmatrix}\) gehöhrt:

\[\begin{aligned}
\vec{d}_1 &= \frac{1}{5}\begin{pmatrix} 21 \\ 9 \end{pmatrix} -  \begin{pmatrix} 6 \\ 3 \end{pmatrix} =  \frac{1}{5} \begin{pmatrix} 21 - 30 \\ 9 - 15  \end{pmatrix} = \frac{1}{5}\begin{pmatrix} -9 \\ -6 \end{pmatrix}, \quad \left| \vec{d}_1 \right| = \sqrt{\frac{81}{25} + \frac{36}{25}}  \\
\vec{d}_3 &= \frac{1}{2}\begin{pmatrix} 15 \\ 9 \end{pmatrix} -  \begin{pmatrix} 6 \\ 3 \end{pmatrix} =  \frac{1}{4}\begin{pmatrix} -3 \\ -3 \end{pmatrix}, \quad \left| \vec{d}_3 \right| = \sqrt{\frac{18}{4}} < \left| \vec{d}_1 \right|
\end{aligned}\]

Die dazugehörige Skizze:

     \textless{}img style=``transform: rotate(90deg);''
src=``./gruppierung.jpg'', width=800,height=600\textgreater{} 

    \hypertarget{aufgabenteil-c}{%
\subsubsection{Aufgabenteil c)}\label{aufgabenteil-c}}

Der Algorithmus konvergiert nach 5 Iterationsschritten (siehe oben). Die
erhaltende Gruppierung war nicht zu erwarten. Man hätte eher erwartet,
dass die drei Punkte links, unten die 6 Punkte und rechts die 3 Punkte
zu einer Domäne zusammentun.

    \hypertarget{aufgabe-20}{%
\subsection{Aufgabe 20}\label{aufgabe-20}}

Noch ein informativer
\href{http://www.cbcity.de/tutorial-neuronale-netze-einfach-erklaert}{Link}
\#\#\# Aufgabenteil a) Die \emph{Lossfunktion} ist ein Maß dafür wie
viel Information/Wahrheit bei zum Beispiel einem Schnitt verloren geht.
Hier bei gibt es verschiedene Defintionen für eine Lossfunktion.
Allgemein soll eine Operation die Lossfunktion minimieren, also die
maximale Information/Wahrheit erhalten. Aus diesem Grund wird die
Lossfunktion dazu verwendet, um z.B. den geeignetsten Schnitt aus einer
Vielzahl von möglichen Schnitten zu ermittelt.

\hypertarget{aufgabenteil-b}{%
\subsubsection{Aufgabenteil b)}\label{aufgabenteil-b}}

Durch Ausprobieren oder durch das analytische Bestimmen des Minimas
(folgen des Gradientens).

\hypertarget{aufgabenteil-c}{%
\subsubsection{Aufgabenteil c)}\label{aufgabenteil-c}}

Die Aktivierungsfunktion steuert, unter Berücksichtigung des Inputs, wie
aktiv das Neuron ist. Hierbei muss der Input auch einen bestimmten
Schwellwert überschreiten, bevor das Neuro überhaupt aktiviert wird.

Wir haben noch drei \textbf{Fragen} bezgl. Neuronen:

\begin{itemize}
\tightlist
\item
  Durch die Aktivierungsfunktion wird somit das erstmal schon mal Noise
  vom Signal getrennt (stimmt das?).
\item
  Die Aktivierungsfunktionen tragen auch zum \emph{lernen} des Netzwerks
  bei (könnt ihr das nochmal erklären?).
\item
  Aktivierungsfunktion der letzten Neuronenebene muss zum gewünschten
  Output passen (Wieso?).
\end{itemize}

Beispiele: - lineare Funktion - Sigmoid Funktion - Leaky ReLU

\hypertarget{aufgabenteil-d}{%
\subsubsection{Aufgabenteil d)}\label{aufgabenteil-d}}

Im biologischen Sinne ist ein Neuron definiert als eine Nervenzelle die
Informationen aufnimmt und verarbeitet.

In der Informatik:

Man verwendete linerare Transformationen \(f=\mathbf{W}x_i + b\), um
Daten zu klassifizieren. Hierbei ist
\(\mathbf{W}\in \mathbb{R}^{N\times M}\) die Dimensionalität der Matrix,
gegeben durch die Anzahl der Eingangsneuronen \(N\) und der
Ausgangsneuronen \(M\). Somit sind \emph{Neuronen} nichts anderes als
die Dimensionalität der lineraten Transformationen. Verbunden werden die
einzelnen Neuronen somit über das Matrixprodukt.

\hypertarget{aufgabenteil-e}{%
\subsubsection{Aufgabenteil e)}\label{aufgabenteil-e}}

In der Bilderkennung werden Neuronalenetzwerke verwendet, weil hier die
hohe Dimensionalität/Komplexität des Eingangssignals verarbeitet und
reduziert werden kann.

Aber auch bei der Klassifizeriung von Datenpunkte, da diese beliebege
dimensionale Fitfunktionen annehmen können und somit Daten geschickt
transformieren und dann fitten.

    \hypertarget{aufgabe-21}{%
\subsection{Aufgabe 21}\label{aufgabe-21}}

    \hypertarget{aufgabenteil-a}{%
\subsubsection{Aufgabenteil a)}\label{aufgabenteil-a}}

\[ \begin{aligned}
\mathrm{dim}(x_i) &= M\times 1, \quad \mathrm{dim}(C) = 1 \times 1, \quad \mathrm{dim}(W) = K \times M,  \quad \mathrm{dim}(b) = K \times 1
\end{aligned}\]

\[\begin{aligned}
\nabla_W \hat{C} = \sum_i^K \underbrace{\frac{\partial\hat{C_i}}{\partial f_{k,i}}}_{\mathrm{dim} = K\times 1}\,\underbrace{\frac{\partial f_{k,i}}{\partial W}}_{{\mathrm{dim} = 1\times M}} \quad \Rightarrow \quad \mathrm{dim}(\nabla_W \hat{C}) = K \times M
\end{aligned}\]

\[\begin{aligned}
\nabla_b \hat{C} = \sum_i^K \underbrace{\frac{\partial\hat{C_i}}{\partial f_{k,i}}}_{\mathrm{dim} = K\times 1}\,\underbrace{\frac{\partial f_{k,i}}{\partial b}}_{{\mathrm{dim} =  K \times 1}} \quad \Rightarrow \quad \mathrm{dim}(\nabla_b \hat{C}) = K \times 1
\end{aligned}\]

    \hypertarget{aufgabenteil-b}{%
\subsubsection{Aufgabenteil b)}\label{aufgabenteil-b}}

\[\begin{aligned}
\nabla_{f_a} \hat{C} &= \frac{1}{m} \sum_i^m \left(-\sum_k^K\nabla_{f_a} \delta(y_i =k) \,\log\left(\frac{\exp(f_{k,i})}{\sum_n \exp(f_{n,i})}\right)\right) \\
&=  -\frac{1}{m} \sum_i^m \delta(y_i =k)\left( \frac{\sum_n \exp(f_{n,i})}{\exp(f_{k,i})\frac{\partial f_{k,i}}{\partial f_{a,i}}}\frac{\exp(f_{k,i}) \frac{\partial f_{k,i}}{\partial f_{a,i}} \sum_j \exp(f_{j,i}) - \exp(f_{k,i})^2 \frac{\partial f_{k,i}}{\partial f_{a,i}}}{\left(\sum_n \exp(f_{n,i})\right)^2} \right) \\
mit \qquad \frac{\partial f_{k,i}}{\partial f_{a,i}}&=\delta_{a,k} \\
&= \frac{1}{m} \sum_i^m \left( \frac{\exp(f_{a,i})}{\sum_n \exp(f_{n,i})} - \delta(y_i =a)\right)
\end{aligned}\]

    \hypertarget{aufgabenteil-c}{%
\subsubsection{Aufgabenteil c)}\label{aufgabenteil-c}}

\[\begin{aligned}
\nabla_W f_{k,i} = \nabla_W \left(W_k \cdot \vec{x_i} + b_k\right) = \begin{pmatrix} \vec{0} & \dots & \vec{0}& \underbrace{\vec{x_i}}_{k_{te}-Spalte} &\vec{0} & \dots & \vec{0} \end{pmatrix}
\end{aligned}\]

Beispiel für

\[\begin{aligned}
W= \begin{pmatrix} a & b & c \\ d & e & f \\ g & h & i \end{pmatrix}, \quad W_2 &= \begin{pmatrix}  d & e & f \end{pmatrix}, \quad \vec{x} = \begin{pmatrix} x_1 & x_2 & x_3 \end{pmatrix}^{\mathrm{T}} \\
\nabla_W f_{2,i} &= \nabla_W \underbrace{\left( dx_1 + ex_2 +fx_3 \right)}_{=:\Gamma} =\begin{pmatrix}  \frac{\partial \Gamma}{\partial a}  & \frac{\partial \Gamma}{\partial d} & \frac{\partial \Gamma}{\partial g} \\ 
\frac{\partial \Gamma}{\partial b} & \frac{\partial \Gamma}{\partial e} & \frac{\partial \Gamma}{\partial h} \\ \frac{\partial \Gamma}{\partial c} & \frac{\partial \Gamma}{\partial f} & \frac{\partial \Gamma}{\partial i} \end{pmatrix} = \begin{pmatrix} 0 & x_1 & 0 \\ 0 & x_2 & 0 \\ 0 & x_3 & 0 \end{pmatrix}
\end{aligned}\]

\[\begin{aligned}
\nabla_b f_{k,i} = \nabla_W \left(W_k \cdot \vec{x_i} + b_k\right) = \begin{pmatrix} 0 \\ \vdots \\ 0 \\ 1 \, \,(k_{te} -Zeile) \\ 0 \\ \vdots \\ 0 \end{pmatrix}
\end{aligned}\]

    \hypertarget{aufgabenteil-d}{%
\subsubsection{Aufgabenteil d)}\label{aufgabenteil-d}}

    \begin{Verbatim}[commandchars=\\\{\}]
{\color{incolor}In [{\color{incolor}2}]:} \PY{n}{pop\PYZus{}1} \PY{o}{=} \PY{n}{pd}\PY{o}{.}\PY{n}{read\PYZus{}hdf}\PY{p}{(}\PY{l+s+s1}{\PYZsq{}}\PY{l+s+s1}{./populationen.hdf5}\PY{l+s+s1}{\PYZsq{}}\PY{p}{,} \PY{n}{key}\PY{o}{=}\PY{l+s+s1}{\PYZsq{}}\PY{l+s+s1}{P\PYZus{}0}\PY{l+s+s1}{\PYZsq{}}\PY{p}{)}
        \PY{n}{pop\PYZus{}2} \PY{o}{=} \PY{n}{pd}\PY{o}{.}\PY{n}{read\PYZus{}hdf}\PY{p}{(}\PY{l+s+s1}{\PYZsq{}}\PY{l+s+s1}{./populationen.hdf5}\PY{l+s+s1}{\PYZsq{}}\PY{p}{,} \PY{n}{key}\PY{o}{=}\PY{l+s+s1}{\PYZsq{}}\PY{l+s+s1}{P\PYZus{}1}\PY{l+s+s1}{\PYZsq{}}\PY{p}{)}
        
        \PY{n}{pop\PYZus{}1}\PY{p}{[}\PY{l+s+s1}{\PYZsq{}}\PY{l+s+s1}{label}\PY{l+s+s1}{\PYZsq{}}\PY{p}{]} \PY{o}{=} \PY{n}{np}\PY{o}{.}\PY{n}{zeros}\PY{p}{(}\PY{n}{pop\PYZus{}1}\PY{o}{.}\PY{n}{shape}\PY{p}{[}\PY{l+m+mi}{0}\PY{p}{]}\PY{p}{)}
        \PY{n}{pop\PYZus{}2}\PY{p}{[}\PY{l+s+s1}{\PYZsq{}}\PY{l+s+s1}{label}\PY{l+s+s1}{\PYZsq{}}\PY{p}{]} \PY{o}{=} \PY{n}{np}\PY{o}{.}\PY{n}{ones}\PY{p}{(}\PY{n}{pop\PYZus{}2}\PY{o}{.}\PY{n}{shape}\PY{p}{[}\PY{l+m+mi}{0}\PY{p}{]}\PY{p}{)}
\end{Verbatim}


    \begin{Verbatim}[commandchars=\\\{\}]
{\color{incolor}In [{\color{incolor}3}]:} \PY{n}{df\PYZus{}pop} \PY{o}{=} \PY{n}{pd}\PY{o}{.}\PY{n}{concat}\PY{p}{(}\PY{p}{[}\PY{n}{pop\PYZus{}1}\PY{p}{,} \PY{n}{pop\PYZus{}2}\PY{p}{]}\PY{p}{,} \PY{n}{ignore\PYZus{}index}\PY{o}{=}\PY{k+kc}{True}\PY{p}{)}
\end{Verbatim}


    \begin{Verbatim}[commandchars=\\\{\}]
{\color{incolor}In [{\color{incolor}4}]:} \PY{k}{def} \PY{n+nf}{soft\PYZus{}max}\PY{p}{(}\PY{n}{x\PYZus{}i}\PY{p}{,} \PY{n}{W}\PY{p}{,} \PY{n}{b}\PY{p}{)}\PY{p}{:}
            \PY{c+c1}{\PYZsh{} Berechne f für alle x gleichzeitig}
            \PY{n}{f} \PY{o}{=} \PY{n}{np}\PY{o}{.}\PY{n}{matmul}\PY{p}{(}\PY{n}{W}\PY{p}{,} \PY{n}{x\PYZus{}i}\PY{p}{[}\PY{p}{[}\PY{l+s+s1}{\PYZsq{}}\PY{l+s+s1}{x}\PY{l+s+s1}{\PYZsq{}}\PY{p}{,}\PY{l+s+s1}{\PYZsq{}}\PY{l+s+s1}{y}\PY{l+s+s1}{\PYZsq{}}\PY{p}{]}\PY{p}{]}\PY{o}{.}\PY{n}{T}\PY{p}{)}\PY{o}{.}\PY{n}{T} \PY{o}{+} \PY{n}{b}
            
            \PY{c+c1}{\PYZsh{} Berchne aus f die Softmax Funktion}
            \PY{n}{q} \PY{o}{=} \PY{p}{(}\PY{n}{np}\PY{o}{.}\PY{n}{exp}\PY{p}{(}\PY{n}{f}\PY{p}{)}\PY{o}{.}\PY{n}{T} \PY{o}{/} \PY{n}{np}\PY{o}{.}\PY{n}{sum}\PY{p}{(}\PY{n}{np}\PY{o}{.}\PY{n}{exp}\PY{p}{(}\PY{n}{f}\PY{p}{)}\PY{p}{,} \PY{n}{axis}\PY{o}{=}\PY{l+m+mi}{1}\PY{p}{)}\PY{o}{.}\PY{n}{T}\PY{p}{)}\PY{o}{.}\PY{n}{T} 
           
            \PY{l+s+sd}{\PYZsq{}\PYZsq{}\PYZsq{} Gebe einen Vektor q zurück, hierbei ist q[0] die zu Klasse 1 zugehörige Softmax funktion }
        \PY{l+s+sd}{        und q[1] bezüglich der Klasse 2}
        \PY{l+s+sd}{    \PYZsq{}\PYZsq{}\PYZsq{}}
            
            \PY{k}{return} \PY{n}{q}
\end{Verbatim}


    \begin{Verbatim}[commandchars=\\\{\}]
{\color{incolor}In [{\color{incolor}5}]:} \PY{k}{def} \PY{n+nf}{indicator\PYZus{}function}\PY{p}{(}\PY{n}{x}\PY{p}{,} \PY{n}{grad\PYZus{}c}\PY{p}{)}\PY{p}{:}
            \PY{c+c1}{\PYZsh{} Subtrahiere von jedem Element was zu Klasse 0 gehört (1, 0) bzw. wenn es zur Klasse 1 gehört (0, 1).}
            \PY{c+c1}{\PYZsh{} Man beachte das ich immer die transponierten Gradienten verwende}
            \PY{n}{grad\PYZus{}c}\PY{p}{[}\PY{n}{x}\PY{p}{[}\PY{l+s+s1}{\PYZsq{}}\PY{l+s+s1}{label}\PY{l+s+s1}{\PYZsq{}}\PY{p}{]}\PY{o}{==} \PY{l+m+mi}{0}\PY{p}{]}\PY{o}{=} \PY{n}{grad\PYZus{}c}\PY{p}{[}\PY{n}{x}\PY{p}{[}\PY{l+s+s1}{\PYZsq{}}\PY{l+s+s1}{label}\PY{l+s+s1}{\PYZsq{}}\PY{p}{]}\PY{o}{==} \PY{l+m+mi}{0}\PY{p}{]} \PY{o}{\PYZhy{}} \PY{n}{np}\PY{o}{.}\PY{n}{array}\PY{p}{(}\PY{p}{[}\PY{l+m+mi}{1}\PY{p}{,}\PY{l+m+mi}{0}\PY{p}{]}\PY{p}{)}
            \PY{n}{grad\PYZus{}c}\PY{p}{[}\PY{n}{x}\PY{p}{[}\PY{l+s+s1}{\PYZsq{}}\PY{l+s+s1}{label}\PY{l+s+s1}{\PYZsq{}}\PY{p}{]}\PY{o}{==} \PY{l+m+mi}{1}\PY{p}{]}\PY{o}{=} \PY{n}{grad\PYZus{}c}\PY{p}{[}\PY{n}{x}\PY{p}{[}\PY{l+s+s1}{\PYZsq{}}\PY{l+s+s1}{label}\PY{l+s+s1}{\PYZsq{}}\PY{p}{]}\PY{o}{==} \PY{l+m+mi}{1}\PY{p}{]} \PY{o}{\PYZhy{}} \PY{n}{np}\PY{o}{.}\PY{n}{array}\PY{p}{(}\PY{p}{[}\PY{l+m+mi}{0}\PY{p}{,}\PY{l+m+mi}{1}\PY{p}{]}\PY{p}{)}
            \PY{k}{return} \PY{n}{grad\PYZus{}c}
\end{Verbatim}


    \begin{Verbatim}[commandchars=\\\{\}]
{\color{incolor}In [{\color{incolor}6}]:} \PY{k}{def} \PY{n+nf}{gradient\PYZus{}cost\PYZus{}function}\PY{p}{(}\PY{n}{x}\PY{p}{,} \PY{n}{soft\PYZus{}max\PYZus{}f}\PY{p}{,} \PY{n}{indicator}\PY{p}{,} \PY{n}{W}\PY{p}{,} \PY{n}{b}\PY{p}{)}\PY{p}{:}
            
            \PY{n}{m} \PY{o}{=} \PY{n}{x}\PY{o}{.}\PY{n}{shape}\PY{p}{[}\PY{l+m+mi}{0}\PY{p}{]}
            \PY{c+c1}{\PYZsh{} Berechne den Softmax funktion}
            \PY{n}{grad\PYZus{}c\PYZus{}i} \PY{o}{=} \PY{n}{soft\PYZus{}max\PYZus{}f}\PY{p}{(}\PY{n}{x}\PY{p}{,} \PY{n}{W}\PY{p}{,} \PY{n}{b}\PY{p}{)}
            \PY{c+c1}{\PYZsh{} Subtrahiere von den jeweiligen Elementen jeweils 1}
            \PY{n}{grad\PYZus{}c\PYZus{}i} \PY{o}{=} \PY{n}{indicator}\PY{p}{(}\PY{n}{x}\PY{p}{,} \PY{n}{grad\PYZus{}c\PYZus{}i}\PY{p}{)}
            \PY{c+c1}{\PYZsh{} Man hätte die obigen Schritte auch in einem zusammen packen können}
            
            \PY{c+c1}{\PYZsh{} Die Matritzenmulptilikation ersetzt die Summation}
            \PY{n}{grad\PYZus{}C\PYZus{}W} \PY{o}{=}\PY{l+m+mi}{1} \PY{o}{/}\PY{n}{m} \PY{o}{*} \PY{n}{np}\PY{o}{.}\PY{n}{matmul}\PY{p}{(}\PY{n}{grad\PYZus{}c\PYZus{}i}\PY{o}{.}\PY{n}{T}\PY{p}{,} \PY{n}{x}\PY{p}{[}\PY{p}{[}\PY{l+s+s1}{\PYZsq{}}\PY{l+s+s1}{x}\PY{l+s+s1}{\PYZsq{}}\PY{p}{,}\PY{l+s+s1}{\PYZsq{}}\PY{l+s+s1}{y}\PY{l+s+s1}{\PYZsq{}}\PY{p}{]}\PY{p}{]}\PY{o}{.}\PY{n}{values} \PY{p}{)}
            \PY{n}{grad\PYZus{}C\PYZus{}b} \PY{o}{=} \PY{l+m+mi}{1} \PY{o}{/} \PY{n}{m} \PY{o}{*} \PY{n}{np}\PY{o}{.}\PY{n}{sum}\PY{p}{(}\PY{n}{grad\PYZus{}c\PYZus{}i}\PY{p}{,} \PY{n}{axis}\PY{o}{=}\PY{l+m+mi}{0}\PY{p}{)}
            
            \PY{k}{return} \PY{n}{grad\PYZus{}C\PYZus{}W}\PY{p}{,}  \PY{n}{grad\PYZus{}C\PYZus{}b}
\end{Verbatim}


    \begin{Verbatim}[commandchars=\\\{\}]
{\color{incolor}In [{\color{incolor}7}]:} \PY{n}{learning\PYZus{}rate} \PY{o}{=} \PY{l+m+mf}{0.5}
        \PY{n}{periodes} \PY{o}{=} \PY{l+m+mi}{100}
        
        \PY{n}{W} \PY{o}{=} \PY{n}{np}\PY{o}{.}\PY{n}{random}\PY{o}{.}\PY{n}{rand}\PY{p}{(}\PY{l+m+mi}{2}\PY{p}{,} \PY{l+m+mi}{2}\PY{p}{)}
        \PY{n}{b} \PY{o}{=} \PY{n}{np}\PY{o}{.}\PY{n}{random}\PY{o}{.}\PY{n}{rand}\PY{p}{(}\PY{l+m+mi}{1}\PY{p}{,} \PY{l+m+mi}{2}\PY{p}{)}
\end{Verbatim}


    \begin{Verbatim}[commandchars=\\\{\}]
{\color{incolor}In [{\color{incolor}8}]:} \PY{n}{grad\PYZus{}C\PYZus{}W}\PY{p}{,} \PY{n}{grad\PYZus{}C\PYZus{}b} \PY{o}{=} \PY{n}{gradient\PYZus{}cost\PYZus{}function}\PY{p}{(}\PY{n}{df\PYZus{}pop}\PY{p}{,} \PY{n}{soft\PYZus{}max}\PY{p}{,} \PY{n}{indicator\PYZus{}function}\PY{p}{,} \PY{n}{W}\PY{p}{,} \PY{n}{b}\PY{p}{)}
\end{Verbatim}


    \hypertarget{aufgabenteil-e}{%
\subsubsection{Aufgabenteil e)}\label{aufgabenteil-e}}

    \begin{Verbatim}[commandchars=\\\{\}]
{\color{incolor}In [{\color{incolor}9}]:} \PY{c+c1}{\PYZsh{} Geradengleichung}
        \PY{k}{def} \PY{n+nf}{gerade}\PY{p}{(}\PY{n}{x}\PY{p}{,} \PY{n}{W}\PY{p}{,} \PY{n}{b}\PY{p}{)}\PY{p}{:}
            \PY{k}{return} \PY{l+m+mi}{1}\PY{o}{/}\PY{p}{(}\PY{n}{W}\PY{p}{[}\PY{l+m+mi}{0}\PY{p}{]}\PY{p}{[}\PY{l+m+mi}{1}\PY{p}{]} \PY{o}{\PYZhy{}} \PY{n}{W}\PY{p}{[}\PY{l+m+mi}{1}\PY{p}{]}\PY{p}{[}\PY{l+m+mi}{1}\PY{p}{]}\PY{p}{)} \PY{o}{*} \PY{p}{(}\PY{p}{(}\PY{n}{W}\PY{p}{[}\PY{l+m+mi}{1}\PY{p}{]}\PY{p}{[}\PY{l+m+mi}{0}\PY{p}{]} \PY{o}{\PYZhy{}} \PY{n}{W}\PY{p}{[}\PY{l+m+mi}{0}\PY{p}{]}\PY{p}{[}\PY{l+m+mi}{0}\PY{p}{]}\PY{p}{)} \PY{o}{*} \PY{n}{x} \PY{o}{+} \PY{n}{b}\PY{p}{[}\PY{l+m+mi}{0}\PY{p}{]}\PY{p}{[}\PY{l+m+mi}{1}\PY{p}{]} \PY{o}{\PYZhy{}} \PY{n}{b}\PY{p}{[}\PY{l+m+mi}{0}\PY{p}{]}\PY{p}{[}\PY{l+m+mi}{0}\PY{p}{]}\PY{p}{)}
\end{Verbatim}


    \begin{Verbatim}[commandchars=\\\{\}]
{\color{incolor}In [{\color{incolor}10}]:} \PY{k}{for} \PY{n}{i} \PY{o+ow}{in} \PY{n+nb}{range}\PY{p}{(}\PY{n}{periodes}\PY{p}{)}\PY{p}{:}
             \PY{n}{grad\PYZus{}C\PYZus{}W}\PY{p}{,} \PY{n}{grad\PYZus{}C\PYZus{}b} \PY{o}{=} \PY{n}{gradient\PYZus{}cost\PYZus{}function}\PY{p}{(}\PY{n}{df\PYZus{}pop}\PY{p}{,} \PY{n}{soft\PYZus{}max}\PY{p}{,} \PY{n}{indicator\PYZus{}function}\PY{p}{,} \PY{n}{W}\PY{p}{,} \PY{n}{b}\PY{p}{)}
             \PY{n}{W} \PY{o}{=} \PY{n}{W} \PY{o}{\PYZhy{}} \PY{n}{learning\PYZus{}rate} \PY{o}{*} \PY{n}{grad\PYZus{}C\PYZus{}W}
             \PY{n}{b} \PY{o}{=} \PY{n}{b} \PY{o}{\PYZhy{}} \PY{n}{learning\PYZus{}rate} \PY{o}{*} \PY{n}{grad\PYZus{}C\PYZus{}b}
\end{Verbatim}


    \begin{Verbatim}[commandchars=\\\{\}]
{\color{incolor}In [{\color{incolor}11}]:} \PY{n}{plt}\PY{o}{.}\PY{n}{scatter}\PY{p}{(}\PY{n}{df\PYZus{}pop}\PY{p}{[}\PY{l+s+s1}{\PYZsq{}}\PY{l+s+s1}{x}\PY{l+s+s1}{\PYZsq{}}\PY{p}{]}\PY{p}{,} \PY{n}{df\PYZus{}pop}\PY{p}{[}\PY{l+s+s1}{\PYZsq{}}\PY{l+s+s1}{y}\PY{l+s+s1}{\PYZsq{}}\PY{p}{]}\PY{p}{,} \PY{n}{c}\PY{o}{=}\PY{n}{df\PYZus{}pop}\PY{p}{[}\PY{l+s+s1}{\PYZsq{}}\PY{l+s+s1}{label}\PY{l+s+s1}{\PYZsq{}}\PY{p}{]}\PY{p}{,} \PY{n}{s}\PY{o}{=}\PY{l+m+mi}{5}\PY{p}{,} \PY{n}{label}\PY{o}{=}\PY{l+s+s1}{\PYZsq{}}\PY{l+s+s1}{Popualtionen}\PY{l+s+s1}{\PYZsq{}}\PY{p}{)}
         
         \PY{n}{plt}\PY{o}{.}\PY{n}{plot}\PY{p}{(}\PY{n}{df\PYZus{}pop}\PY{p}{[}\PY{l+s+s1}{\PYZsq{}}\PY{l+s+s1}{x}\PY{l+s+s1}{\PYZsq{}}\PY{p}{]}\PY{p}{,} \PY{n}{gerade}\PY{p}{(}\PY{n}{df\PYZus{}pop}\PY{p}{[}\PY{l+s+s1}{\PYZsq{}}\PY{l+s+s1}{x}\PY{l+s+s1}{\PYZsq{}}\PY{p}{]}\PY{p}{,} \PY{n}{W}\PY{p}{,} \PY{n}{b}\PY{p}{)}\PY{p}{,} \PY{n}{label}\PY{o}{=}\PY{l+s+s1}{\PYZsq{}}\PY{l+s+s1}{Trenngerade}\PY{l+s+s1}{\PYZsq{}}\PY{p}{)}
         
         \PY{n}{plt}\PY{o}{.}\PY{n}{xlabel}\PY{p}{(}\PY{l+s+s1}{\PYZsq{}}\PY{l+s+s1}{x}\PY{l+s+s1}{\PYZsq{}}\PY{p}{)}
         \PY{n}{plt}\PY{o}{.}\PY{n}{ylabel}\PY{p}{(}\PY{l+s+s1}{\PYZsq{}}\PY{l+s+s1}{y}\PY{l+s+s1}{\PYZsq{}}\PY{p}{)}
         \PY{n}{plt}\PY{o}{.}\PY{n}{legend}\PY{p}{(}\PY{p}{)}
\end{Verbatim}


\begin{Verbatim}[commandchars=\\\{\}]
{\color{outcolor}Out[{\color{outcolor}11}]:} <matplotlib.legend.Legend at 0x7f3c52de3b00>
\end{Verbatim}
            
    \begin{center}
    \adjustimage{max size={0.9\linewidth}{0.9\paperheight}}{output_22_1.png}
    \end{center}
    { \hspace*{\fill} \\}
    

    % Add a bibliography block to the postdoc
    
    
    
    \end{document}
